\hypertarget{spillover-sub-model-matrof}{%
\section{Spillover sub-model MATROF}\label{spillover-sub-model-matrof}}

Surface runoff and groundwater runoff are calculated using a simplified
version of TOPMODEL (Beven and Kirkby, 1979).

\hypertarget{overview-of-topmodel}{%
\subsection{Overview of TOPMODEL}\label{overview-of-topmodel}}

In the TOPMODEL, we consider the horizontal distribution of groundwater
level along the slope in the basin. It is assumed that the groundwater
flow down a slope is balanced with the total recharge rate of the
groundwater recharge rate at the slope above the point (quasi-stationary
assumption). Then, the lower the slope is, the larger the groundwater
flow should be. Based on another assumption, which will be discussed
later, it is assumed that the shallow groundwater surface is necessary
for the groundwater flow to be high. Thus, the distribution of shallow
groundwater level is derived for the lower slopes. When the average
groundwater level is shallower than a certain level, the groundwater
level rises to the ground below a certain point of the slope and forms a
saturated area. Thus, TOPMODEL is characterized by the fact that the
concepts of mean groundwater level, saturated area, and velocity of
groundwater flow, which are important for the estimation of runoff, are
physically consistent with each other.

In the TOPMODEL, three main assumptions are made as follows

The saturated hydraulic conductivity of soil decays exponentially with
depth.

The slope of the groundwater surface is nearly equal to the slope slope
locally.

The groundwater flow down a slope corresponds to the accumulated
groundwater recharge rate above the point of slope.

In the following, the use of symbols follows the conventions of TOPMODEL
(Sivapalan et al., 1987 ; Stieglitz et al., 1997).

The assumption 1 can be written as

\begin{eqnarray}
 K_s(z) = K_0 \exp (-f z)
\end{eqnarray}

\(K_s(z)\) is the saturated hydraulic conductivity of soil at the depth
\(z\), \(K_0\) is the saturated hydraulic conductivity at the ground
surface, and \(f\) is the attenuation coefficient.

If the depth of the groundwater surface at a certain point (\(i\)) is
defined as \(z_i\), then the groundwater flux down slope at that point
(\(q_i\)) is expressed as

\begin{eqnarray}
 q_i = \int_{z_i}^Z K_s(z) dz \cdot \tan\beta
   = \frac{K_0}{f}  \tan\beta [\exp(-f z_i) - \exp(-f Z)]
\end{eqnarray}

\(\beta\) is the slope slope slope, using assumption 2. Although \(Z\)
is the depth of the impermeable surface, the term \(Z\) is usually
assumed to be deeper than that of \(1/f\), and the term \(\exp(-f Z)\)
is omitted. The slope directional soil water flux in the unsaturated
zone above the groundwater surface is neglected because it is small.

Assuming that the groundwater recharge rate is uniformly set to the
\(R\), Assumption 3 can be expressed as follows

\begin{eqnarray}
 a R = \frac{K_0}{f} \tan\beta \exp(-f z_i)
\end{eqnarray}

Here, \(a\) is the total upstream area (per unit contour length at the
point \(i\)) relative to the point \(i\).

Solving this for \(z_i\) yields the following.

\begin{eqnarray}
 z_i = -\frac{1}{f} \ln \left( \frac{faR}{K_0 \tan \beta}\right)
\end{eqnarray}

The average groundwater depth in the region (\(A\)) is the depth of the
groundwater table (\(\overline{z}\)),

\begin{eqnarray}
   \overline{z} = \frac1{A}\int_{A} z_i dA
  = - \Lambda - \frac1{f} \ln R
\end{eqnarray}

\begin{eqnarray}
 \Lambda \equiv
  \frac1{A}\int_{A} \ln \left( \frac{fa}{K_0 \tan \beta}\right) dA
\end{eqnarray}

Thus, the recharge rate (\(R\)) as a function of mean groundwater depth
(\(\overline{z}\)) is expressed as

\begin{eqnarray}
 R = \exp (-f \overline{z} -\Lambda)
\end{eqnarray}

According to Assumption 3, this is the amount of groundwater discharged
from the area (\(A\)).

Substituting \(R\) into (4) gives the following relationship between
\(z_i\) and \(\overline{z}\).

\begin{eqnarray}
 z_i = \overline{z} - \frac{1}{f} \left[
\ln \left( \frac{fa}{K_0 \tan \beta}\right) - \Lambda
\right]
\end{eqnarray}

The region satisfying the \(z_i \leq 0\) is the surface saturation
region.

\hypertarget{application-of-topmodel-under-the-assumption-of-simplified-terrain}{%
\subsection{Application of TOPMODEL under the assumption of simplified
terrain}\label{application-of-topmodel-under-the-assumption-of-simplified-terrain}}

When TOPMODEL is used, detailed topographical data of the target area
are usually required, but here we estimate roughly the average shape of
the slopes in the grid based on the data of average slope and standard
deviation of the elevation of the grid (this method is currently
provisional and requires further study).

The topography of the grid is represented by the slope of \(\beta_s\)
with a uniform slope of \(\beta_s\) and the distance from the ridge to
the valley of \(L_s\).

\(L_s\) is estimated using the standard deviation of elevation
(\(\sigma_z\)) as follows.

\begin{eqnarray}
 L_s = 2\sqrt{3} \sigma_z / \tan\beta_s
\end{eqnarray}

\(2\sqrt{3}\sigma_z\) is the difference between the elevation of the
ridge and the valley in the serrated terrain with the standard deviation
of the elevation of the \(\sigma_z\).

Taking the \(x\) axis from the ridge to the valley on the horizontal
plane. Since the total upstream area of the curve at the \(x\) is \(x\),
(4) becomes the following.

\begin{eqnarray}
 z(x) = - \frac{1}{f} \ln \left( \frac{fxR}{K_0 \tan \beta_s}\right)
\end{eqnarray}

Based on this, the mean groundwater surface is calculated from (5) as

\begin{eqnarray}
 \overline{z} = \frac 1{L_s}\int_0^{L_s} z(x) dx
 = - \frac1{f}\left[
 \ln \left( \frac{f L_s R}{K_0 \tan\beta_s}\right) -1
\right]
\end{eqnarray}

Groundwater recharge rate is from (7)

\begin{eqnarray}
 R = \frac{K_0 \tan\beta_s}{f L_s}\exp(1-f \overline{z})
\end{eqnarray}

The relationship between groundwater level and mean groundwater level in
\(x\) is from (8)

\begin{eqnarray}
 z(x) = \overline{z} - \frac{1}{f}\left(
\ln \frac{x}{L_s} + 1
\right)
\end{eqnarray}

The result is Solving for \(z(x) \leq 0\) with respect to \(x\) yields
the following.

\begin{eqnarray}
 x \geq x_0
\end{eqnarray}

\begin{eqnarray}
x_0 = L_s \exp(f\overline{z}-1)
\end{eqnarray}

Therefore, the area factor of the saturation region is

\begin{eqnarray}
 A_{sat} = (L_s - x_0)/ L_s = 1 - \exp(f\overline{z}-1)
\end{eqnarray}

In the case of \(A_{sat} \geq 0\) and \(\overline{z} > 1/f\), the
saturation region does not exist. However, for the \(A_{sat} \geq 0\)
and \(\overline{z} > 1/f\), there is no saturation region.

\hypertarget{calculation-of-flow-rate}{%
\subsection{Calculation of flow rate}\label{calculation-of-flow-rate}}

Four discharge mechanisms are considered and the total amount of
discharge by each mechanism is defined as the total amount of discharge
from the grid.

\begin{eqnarray}
 Ro = Ro_s + Ro_i + Ro_o + Ro_b
\end{eqnarray}

\(Ro_s\) is a saturation excess runoff (Dunne runoff), \(Ro_i\) is an
infiltration excess runoff (Horton runoff), \(Ro_o\) is an overflow of
the first layer of soil, and the above is the total amount of runoff
from the grid by each mechanism. Classified. \(Ro_b\) is a groundwater
runoff .

\hypertarget{estimates-of-mean-groundwater-depth.}{%
\subsubsection{Estimates of Mean Groundwater
Depth.}\label{estimates-of-mean-groundwater-depth.}}

Assuming that soil moisture content starts from the lowest layer of soil
and that the layer that becomes unsaturated for the first time is the
\(k_{WT}\)th layer, the average depth of the groundwater table
(\(\overline{z}\)) is estimated as follows

\begin{eqnarray}
 \overline{z} = z_{g(k_{WT}-1/2)} - \psi_{k_{WT}}
\end{eqnarray}

This corresponds to the assumption that the moisture potential at the
top of the unsaturated layer is set to \(\psi_{k_{WT}}\), under which
the distribution of soil moisture is considered to be in equilibrium
(i.e., the equilibrium condition between gravity and capillary force).

If the \(\overline{z} > z_{g(k_{WT}+1/2)}\) is the lowest level, no
groundwater surface is assumed to exist in the \(k_{WT}\) region. If the
\(k_{WT}\) is not the bottom layer, the uppermost layer of the saturated
groundwater layer (the uppermost layer) is assumed to be the \(k_{WT}\)
and the above equation is applied to the lower layer.

In the case where the frozen ground surface exists in the middle of the
soil, the depth of the groundwater surface is estimated above the frozen
ground surface.

\hypertarget{calculation-of-groundwater-runoff}{%
\subsubsection{Calculation of Groundwater
Runoff}\label{calculation-of-groundwater-runoff}}

Since the groundwater discharge is equal to the recharge rate of (12)
under quasi-steady assumptions, we can assume that the groundwater
recharge rate is the same as that of (12),

\begin{eqnarray}
 Ro_b = \frac{K_0 \tan\beta_s}{f L_s}\exp(1-f \overline{z})
\end{eqnarray}

(1). However, in the case of a frozen surface under the ground surface,
see the case where the term \(\exp(-fZ)\) in (2) is not omitted,

\begin{eqnarray}
 Ro_b = \frac{K_0 \tan\beta_s}{f L_s}
  [ \exp(1-f \overline{z}) - \exp(1-f z_f) ]
\end{eqnarray}

The depth of the frozen ground surface is defined as \(z_f\) is the
depth of the frozen ground. In this case, the other formulas of the
TOPMODEL system should be different, but we do not change them for the
sake of simplicity.

If there is an antifreeze layer beneath the frozen ground surface and a
groundwater surface exists there, the runoff of groundwater is
calculated and added in the same way.

The groundwater runoff is later removed from the \(k_{WT}\) layer.

\begin{eqnarray}
 Ro_{(k_{WT})} = Ro_b
\end{eqnarray}

\(Ro_{(k)}\) represents the runoff flux from the soil in the \(k\)th
layer.

\hypertarget{calculation-of-surface-runoff.}{%
\subsubsection{Calculation of Surface
Runoff.}\label{calculation-of-surface-runoff.}}

All precipitation that falls in the saturated area of the earth's
surface will runoff intact (saturation excess runoff).

\begin{eqnarray}
 Ro_s = (Pr_c^{**} + Pr_l^{**}) A_{sat}
\end{eqnarray}

The area fraction of saturated area \(A_{sat}\) is given by (16). Here,
the correlation between precipitation distribution on the subgrid and
the topography is ignored.

The precipitation that falls in the unsaturated area is infiltrated
excess runoff (infiltration excess runoff). The soil infiltration
capacity is given in terms of the saturated hydraulic conductivity of
the first layer of soil for simplicity. Convective precipitation is
assumed to be localized, and the area fraction of the precipitation area
(\(A_c\)) is assumed to be uniform (the standard value is 0.1).
Stratified precipitation is assumed to be uniform.

\begin{eqnarray}
 Ro_i^c = \max( Pr_c^{**}/A_c + Pr_l^{**} - K_{s(1)}, 0 ) (1 - A_{sat}) \\
 Ro_i^{nc} = \max( Pr_l^{**} - K_{s(1)}, 0 ) (1 - A_{sat})
\end{eqnarray}

\begin{eqnarray}
 Ro_i = A_c Ro_i^c + ( 1 - A_c ) Ro_i^{nc}
\end{eqnarray}

\(Ro_i^c\) and \(Ro_i^{nc}\) are \(Ro_i\) for convective and
non-convective precipitation areas, respectively, and \(K_{s(1)}\) is
the saturated hydraulic conductivity of the first soil layer.

The overflow of the first layer of soil allows for a small amount of
waterlogging \(w_{str}\) (1 mm by default),

\begin{eqnarray}
 Ro_o = \max(w_{(1)} - w_{sat(1)} - w_{str}, 0) \rho_w \Delta z_{g(1)} / \Delta t_L
\end{eqnarray}

This amount will be subtracted from the first layer of soil later. This
amount will later be subtracted from the first layer of soil, and is
therefore included in the runoff from the first layer.

\begin{eqnarray}
 Ro_{(1)} = Ro_{(1)} + Ro_o
\end{eqnarray}

\hypertarget{water-flux-to-the-soil}{%
\subsection{Water flux to the soil}\label{water-flux-to-the-soil}}

The water fluxes fed to the soil through the runoff process are as
follows.

\begin{eqnarray}
 Pr^{***} = Pr^{**}_c + Pr^{**}_l - Ro_s - Ro_i
\end{eqnarray}

Translated with www.DeepL.com/Translator (free version)
