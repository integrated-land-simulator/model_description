\section{はじめに}
Minimal Advanced Treatments of Surface Interaction and RunOff
(MATSIRO) は, 全球気候モデル CCSR/NIES AGCM, その他のモデルに導入すること
を目的に開発された陸面過程パラメタリゼーションである.
MATSIRO は, 主としてグリッドスケール数10km 以上の大気モデルと結合して, 1ヵ
月から数100年の長期間積分を行うような気候計算に用いられることを目的とし
て設計されている.
大気陸面系の水・エネルギー循環過程の中で, このような時空間スケールにおい
て考慮されるべき過程ができる限り適切に表現でき, しかも結果の解釈が容易で
あるようにそれらができる限り簡潔にモデル化されることを開発の上での留意点
とした.
MATSIRO の開発は, CCSR/NIES AGCM5.4g の陸面サブモデルをベースに,
Watanabe(1994)の植生キャノピーパラメタリゼーションを結合し, 同時に積雪,
流出などの過程を改良することにより行われた.
その後, AGCM の構成の変更に伴い, フラックスカプラーへの対応, 並列化対応
を経て, 現在の AGCM5.6 に対応したものになっている.
植生の生理過程としては, 初期には気孔抵抗に Jarvis 型の関数を用いていたが,
近年の気候−生態系相互作用研究の進展により世界中で準標準的に用いられるよ
うになった Farquhar 型の光合成スキームを SiB2 のコードから移植した.

\section{MATSIRO の構成}

\subsection{MATSIRO 全体}

\subsubsection{全体の構成}

MATSIRO の構成は,
\begin{itemize}
\item フラックス計算部 (LNDFLX)
\item 陸面積分部 (LNDSTP)
\end{itemize}
の二つに分かれる.
これは, AGCM5.6 の標準陸面過程ルーチンと同様である.

フラックス計算部は, 大気モデルの物理過程の一部として, 大気モデルの時間ス
テップで呼ばれる.
これに対して, 陸面積分部は, 大気モデルや海洋モデルと対等な陸面モデルとし
て, 結合モデルメインルーチンから呼ばれる. 時間ステップは大気モデルとは独
立に設定される.

フラックス計算部と陸面積分部の間のデータのやり取りは, フラックスカプラー
を通して行われる.
フラックス計算部では, グリッド内の積雪面と無雪面を別々に取扱い, それぞれ
についての地表フラックスを求める. これを積雪面と無雪面の面積比で重み付け
平均したものが, フラックスカプラーを通じて陸面積分部に渡される.
この際, 時間的な平均操作も同時に行われる.

プログラム上は, ドライバープログラム (matdrv.F) の中にフラックス計算部の
エントリ LNDFLX と陸面積分部のエントリ LNDSTP があり, それぞれから必要な
諸過程のサブルーチンが呼ばれる. LNDFLX と LNDSTP は MATSIRO の内部変数を
共有する.

\subsubsection{MATSIRO の内部変数}

MATSIRO は内部変数として以下を持つ.

\begin{tabular}{llll}
$T_{s(l)}$      & $(l=1,2)$           & 地表温度 & [K]\\
$T_{c(l)}$      & $(l=1,2)$           & キャノピー温度 & [K]\\
\\
$T_{g(k)}$      & $(k=1,\ldots,K_g)$  & 土壌温度 & [K]\\
$w_{(k)} $      & $(k=1,\ldots,K_g)$  & 土壌水分量 & [m$^3$/m$^3$]\\
$w_{i(k)}$      & $(k=1,\ldots,K_g)$  & 凍結土壌水分量 & [m$^3$/m$^3$]\\
\\
$w_c$         &                     & キャノピー上水分量 & [m]\\
$Sn$          &                     & 積雪量 & [kg/m$^2$]\\
$T_{Sn(k)}$      & $(k=1,\ldots,K_{Sn})$  & 積雪温度 & [K]\\
$\alpha_{Sn(b)}$ & $(b=1,2,3)$         & 積雪アルベド & [$-$]
\end{tabular}
\medskip

ここで, $l=1,2$ はそれぞれ無雪域と積雪域, $k$ は土壌や積雪の鉛直層番号
(最上層が 1 で, 下に向かって増える), $K_g$ は土壌層数, $K_{Sn}$ は積雪層数,
$b=1,2,3$ はそれぞれ可視, 近赤外, 赤外の波長帯を表す.

標準では土壌は5層であり, 各層の厚さは上から 5cm, 20cm, 75cm, 1m, 2m に
とる. 土壌温度, 土壌水分量および凍結土壌水分量の定義点は同じである.

積雪は可変層数で, 積雪量の増加に伴い層数を増やす. 標準では最大3層である.

地表温度とキャノピー温度は熱容量を持たない, いわゆる skin 温度であるが,
形式的に予報変数である (今の計算方法では前のステップの値で評価した安定度
などを使うため, 前のステップの値に依存する. もしも更新した値で安定度など
を評価して収束するまで繰り返し計算を行うとすれば, 前のステップの値に依存
しない完全な診断変数になる). その他の変数はいずれも前のステップの値を必
ず必要とする予報変数である.

地表温度とキャノピー温度はフラックス計算部において更新される.
それ以外の変数(本来の予報変数)は, 全て陸面積分部において更新される.

\subsection{フラックス計算部}

\subsubsection{フラックス計算部の構成}

フラックス計算部は以下の手順で進む.
\begin{enumerate}
 \item 入力変数をカプラーから受け取る.
 \item サブグリッドの積雪面積率を診断する.
 \item 無雪域($l=1$)と積雪域($l=2$)のそれぞれについて, 諸過程のサブルー
       チンを呼び出し, フラックスの計算を行うとともに地表温度とキャノピー
       温度を更新する. 具体的には, 以下のサブルーチンが順に呼ばれる.
   \begin{enumerate}
    \item MATLAI\ \ 植生形状パラメタ(LAI, 植生高)のセット
    \item MATRAD\ \ 放射パラメタ(アルベド, 植生透過率など)の計算
    \item MATBLK\ \ 乱流パラメタ(バルク係数)の計算 (運動量および熱)
    \item MATRST\ \ 気孔抵抗, 裸地面蒸発抵抗などの計算
    \item MATBLQ\ \ 乱流パラメタ(バルク係数)の計算 (水蒸気)
    \item MATFLX\ \ 地表フラックスの計算
    \item MATGHC\ \ 地中熱伝導の計算
    \item MATSHB\ \ 地表面・キャノピーのエネルギーバランスを解く
   \end{enumerate}
 \item 出力変数をカプラ−へ渡す.
 \item ヒストリ出力変数の登録と足しこみを行う.
\end{enumerate}

(a)〜(e) は境界値サブモデルに分類され, プログラムは matbnd.F にまとめら
れている.

(f)〜(h) は地表サブモデルに分類され, プログラムは matsfc.F にまとめられ
ている.

(d)から光合成スキーム MATPHT が呼ばれる. 光合成スキームは matpht.F にま
とめられている.

\subsubsection{フラックス計算部の入力変数}

フラックス計算部へは以下の変数が入力される.

\begin{tabular}{llll}
$u_a$          &                     & 大気第1層東西風 & [m/s] \\
$v_a$          &                     & 大気第1層南北風 & [m/s] \\
$T_a$          &                     & 大気第1層気温   & [K]   \\
$q_a$          &                     & 大気第1層比湿   & [kg/kg]\\
$P_a$          &                     & 大気第1層気圧   & [Pa]  \\
$P_s$          &                     & 地表気圧        & [Pa]  \\
\\
$R^{\downarrow}_{(d,b)}$ & $(d=1,2;b=1,2,3)$ & 地表下向き放射フラックス & [W/m$^2$]\\
$\cos\zeta$    &                     & 太陽天頂角の余弦 & [$-$]\\
\end{tabular}
\medskip

ここで, $d=1,2$ はそれぞれ直達と散乱, $b=1,2,3$ はそれぞれ可視, 近赤外, 赤外の波長帯を表す.

\subsubsection{フラックス計算部の出力変数}

フラックス計算部からは以下の変数が出力される.

\begin{tabular}{llll}
$\tau_x$       &                     & 地表東西風ストレス & [N/m$^2$] \\
$\tau_y$       &                     & 地表南北風ストレス & [N/m$^2$] \\
$H$            &                     & 地表顕熱フラックス & [W/m$^2$] \\
$E$            &                     & 地表水蒸気フラックス & [kg/m$^2$/s] \\
$R^{\uparrow}_S$ &                  & 地表上向き短波放射フラックス & [W/m$^2$]\\
$R^{\uparrow}_L$ &                  & 地表上向き長波放射フラックス & [W/m$^2$]\\
$\alpha_{s(b)}$ & $(b=1,2,3)$         & 地表アルベド   & [$-$] \\
$T_{sR}$      &                     & 地表放射温度   & [K]   \\
$F_{g(1/2)}$      &                     & 地表熱伝導フラックス & [W/m$^2$] \\
$F_{Sn(1/2)}$     &                     & 積雪面熱伝導フラックス & [W/m$^2$] \\
$Et_{(i,j)}$   & $(i=1,2;j=1,2,3)$   & 蒸発散各成分   & [kg/m$^2$/s] \\
$\Delta F_{conv}$&                     & 地表面エネルギー収束 & [W/m$^2$] \\
$F_{root(k)}$      & $(k=1,\ldots,K_g)$  & 根の吸い上げフラックス & [kg/m$^2$/s] \\
$LAI$          &                     & 葉面積指数 & [m$^2$/m$^2$] \\
$A_{Snc}$     &                     & キャノピー凍結面積率 & [$-$] \\
\end{tabular}
\medskip

ここで, 蒸発散の $i=1,2$ はそれぞれ液体および固体を, $j=1,2,3$ はそれぞ
れ裸地面(林床)蒸発, 蒸散, キャノピー上水分の蒸発を表す. それ以外の添え字
については前出と同じである.

\subsection{陸面積分部}

\subsubsection{陸面積分部の構成}

陸面積分部は以下の手順で進む.
\begin{enumerate}
 \item 入力変数をカプラーから受け取る.
 \item 諸過程のサブルーチンを呼び出し, 陸面予報変数を更新する. 具体的に
       は, 以下のサブルーチンが順に呼ばれる.
   \begin{enumerate}
    \item MATCNW\ \ キャノピーの水収支の計算
    \item MATSNW\ \ 積雪量, 積雪温度, 積雪アルベドの計算
    \item MATROF\ \ 流出量の計算
    \item MATGND\ \ 土壌温度, 土壌水分, 凍土の計算
   \end{enumerate}
 \item 出力変数をカプラ−へ渡す.
 \item ヒストリ出力変数の登録と足しこみを行う.
\end{enumerate}

陸面積分部のメインから呼ばれる各サブルーチンは, それぞれが一つのサブモデ
ルを構成する. 各サブモデルのプログラムは, それぞれが1つのファイルにまと
められている. 具体的には以下のようである.

\begin{itemize}
 \item MATCNW (matcnw.F) キャノピー水収支サブモデル
 \item MATSNW (matsnw.F) 積雪サブモデル
 \item MATROF (matrof.F) 流出サブモデル
 \item MATGND (matgnd.F) 土壌サブモデル
\end{itemize}

サブモデルは基本的にこの順に実行されるが, 必要に応じて, 他のサブモデルの
管理するパラメータの値を参照するために, サブモデル間でサブルーチンを呼び
合うことがある.
また, フラックス計算部から, 同様の目的で上記サブモデル内のサブルーチンが
呼ばれることがある.

\subsubsection{陸面積分部の入力変数}

陸面積分部へは以下の変数が入力される.

\begin{tabular}{llll}
$Pr_{c}$       &                     & 対流性降雨フラックス & [kg/m$^2$/s] \\
$Pr_{l}$       &                     & 層状性降雨フラックス & [kg/m$^2$/s] \\
$P_{Snc}$      &                     & 対流性降雪フラックス & [kg/m$^2$/s] \\
$P_{Snl}$      &                     & 層状性降雪フラックス & [kg/m$^2$/s] \\
$F_{g(1/2)}$      &                     & 地表熱伝導フラックス & [W/m$^2$] \\
$F_{Sn(1/2)}$     &                     & 積雪面熱伝導フラックス & [W/m$^2$] \\
$Et_{(i,j)}$   & $(i=1,2;j=1,2,3)$   & 蒸発散各成分   & [kg/m$^2$/s] \\
$\Delta F_{conv}$&                     & 地表面エネルギー収束 & [W/m$^2$] \\
$F_{root(k)}$      & $(k=1,\ldots,K_g)$  & 根の吸い上げフラックス & [kg/m$^2$/s] \\
$LAI$          &                     & 葉面積指数 & [m$^2$/m$^2$] \\
$A_{Snc}$     &                     & キャノピー凍結面積率 & [$-$] \\
\end{tabular}
\medskip

\subsubsection{陸面積分部の出力変数}

陸面積分部からは以下の変数が出力される.

\begin{tabular}{llll}
$Ro$           &                     & 流出量             & [kg/m$^2$/s] \\
\end{tabular}
\medskip

流出量は, 河道網モデルの入力変数として使われる.

\subsection{外部パラメタ}

MATSIRO の実行に必要な外部パラメタは, 各グリッドのパラメタ値を水平分布
(マップ)によって与えるものと, 土地被覆タイプごともしくは土壌タイプごとの
パラメタ値をテーブルによって与えるものの2種類に大別される. 土地被覆タイ
プと土壌タイプはマップにより与えられるパラメタの一つであり, これを通じて,
テーブルにより与えられる各パラメタが各グリッドに割り当てられる. すなわち,
\begin{quote}
 マップにより与えられるパラメタ
 \[
 \phi(i,j)
 \]
 テーブルにより与えられるパラメタ
 \[
 \psi(I), \ \ I = I_L(i,j)
 \]
もしくは
 \[
 \psi(I), \ \ I = I_S(i,j)
 \]
\end{quote}
のようである. ここで, $(i,j)$ はグリッドの水平位置のインデックス, $I_L$
は土地利用タイプ, $I_S$ は土壌タイプである.

\subsubsection{マップにより与えられる外部パラメタ}

マップにより与えられる外部パラメタの種類は以下のとおりである.

\begin{tabular}{lllll}
 $I_L$ & & 土地被覆タイプ   & 定数 & [-] \\
 $I_S$ & & 土壌タイプ       & 定数 & [-] \\
 $LAI_0$ & & 葉面積指数 (LAI) & 月毎 & [m$^2$/m$^2$] \\
 $\alpha_{0(b)}$ & $(b=1,2,3)$ & 地表面(林床)アルベド & 定数 & [-] \\
 $\tan\beta_{s}$ & & 地表平均傾斜の正接 & 定数 & [-] \\
 $\sigma_z$ & & 標高標準偏差 & 定数 & [m]
\end{tabular}

\subsubsection{土地被覆タイプごとのテーブルにより与えられる外部パラメタ}

土地被覆タイプごとのテーブルにより与えられる外部パラメタは以下のとおりで
ある.

\begin{tabular}{llll}
$h_0$ & & 植生高 & [m] \\
$h_{B0}$ & & キャノピー底面の高さ & [m] \\
$r_{f(b)}$ & ($b$=1,2) & 個葉の反射率 & [-] \\
$t_{f(b)}$ & ($b$=1,2) & 個葉の透過率 & [-] \\
$f_{root(k)}$ & ($k=1,\ldots,K_g$) & 根の存在比率 & [-] \\
$c_d$ & & 個葉と大気との運動量交換係数 & [-] \\
$c_h$ & & 個葉と大気との熱交換係数 & [-] \\
$f_V$ & & 植生被覆率 & [-] \\
$V_{\max}$ & & Rubisco 反応容量 & [m/s] \\
$m$ & & $A_n$--$g_s$ 関係の傾き & [-] \\
$b$ & & $A_n$--$g_s$ 関係の切片 & [m/s] \\
$\epsilon_3$, $\epsilon_4$ & & 光子あたりの光合成効率 & [m/s/mol] \\
$\theta_{ce}$ & & $w_c$ と $w_e$ の結合係数 & [-] \\
$\theta_{ps}$ & & $w_p$ と $w_s$ の結合係数 & [-] \\
$f_d$ & & 呼吸係数 & [-] \\
$s_2$ & & 高温抑制の臨界温度 & [K] \\
$s_4$ & & 低温抑制の臨界温度 & [K]
\end{tabular}

\subsubsection{土壌タイプごとのテーブルにより与えられる外部パラメタ}

土地被覆ごとのテーブルにより与えられる外部パラメタは以下のとおりである.

\begin{tabular}{llll}
$c_{g(k)}$ & ($k=1,\ldots,K_g$) & 土壌の比熱 & [J/m$^3$] \\
$k_{g(k)}$ & ($k=1,\ldots,K_g$) & 土壌の熱伝導率 & [W/m/K] \\
$w_{sat(k)}$ & ($k=1,\ldots,K_g$) & 土壌の空隙率 & [m$^3$/m$^3$] \\
$K_{s(k)}$ & ($k=1,\ldots,K_g$) & 土壌の飽和透水係数 & [m/s] \\
$\psi_{s(k)}$ & ($k=1,\ldots,K_g$) & 土壌の飽和水分ポテンシャル & [m] \\
$b_{(k)}$ & ($k=1,\ldots,K_g$) & 土壌の水分ポテンシャル曲線の指数 & [-]
\end{tabular}
